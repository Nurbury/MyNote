\section{C语言基础}

\subsection{标识符和关键字}
C语言以数字、字母、下画线组成。其中以字母和下画线作为开头。

\begin{table}
    \centering
    \caption{数据类型关键字}
    \begin{tabular}{ll}
        \toprule
        关键字 & 描述 \\
		\midrule
        char & 声明字符变量 \\
        int & 声明整型变量 \\
        float & 声明单精度浮点数变量 \\
        double & 声明双精度浮点数变量\\
        long & 声明长整数变量 \\
        short & 声明短整数变量 \\
        void & 声明无返回值类型 \\
		\bottomrule
	\end{tabular}
\end{table}


\subsection{运算符}

\begin{itemize}
	\item 算术运算符 +、-、*、/、\%
	\item 关系运算符 ==、!=、>、<、>=、<=
	\item 逻辑运算符 \&\&、||、!
	\item 位运算符 \&、|、~、\^、>>、<<
	\item 赋值运算符 =、+=、-=、*=、/=、%=
\end{itemize}



\subsection{数据类型}

\begin{table}
	\centering
	\begin{tabular}{ll}
		\toprule
		类型 && 存储大小 && 值范围 \\
		\midrule
		char && 1byte && -128~127 或 0~255 \\
		unsigned char && 1byte && 0~255 \\
		signed char && 1byte && -128~127 \\
		int && 2bytes or 4bytes && -32768~32767 \\
		unsigned int && 2bytes or 4bytes && 0~65535 \\
		signed int && 2bytes or 4bytes && -32768~32767 \\
		short && 2bytes && -32768~32767 \\
		unsigned short && 2bytes && 0~65535 \\
		long && 4bytes && -2147483648~3147483247 \\
		unsigned long && 4bytes && 0~4294967295 \\
		\midrule
		float && 4bytes && 1.2E-38 ~ 3.4E+38 \\
		double && 8bytes && 2.3E-308 ~ 1.7E+308 \\
		long double && 16bytes && 3.4E-4932 ~ 1.1E+4932 \\
		\bottomrule
	\end{tabular}
\end{table}

\subsubsection{类型转换}
类型转换是将一个数据类型的值转换为另外一个数据类型。

\begin{itemize}
	\item 隐式类型转换: 自动类型转换,通常是将小的数据类型转换为大的数据类型。
	\item 显式类型转换: 该类型转换需要使用强制类型转换符,将一种类型的值强制转换为另一种类型。
\end{itemize}


\subsection{常量 和 变量}
常量是固定值,可以是整型常量、浮点型常量、字符常量、枚举常量

\noindent
\begin{minipage}{\linewidth}
\begin{lstlisting}[
	language=java,
	caption={常量设置},
	xleftmargin=20pt,
]
const int MAX = 100;
const float PI = 3.14;
const char NEWLINE = '\n';

\end{lstlisting}
\end{minipage}

变量是可以变化的,



	


	
        
