\section{Planning}
大部分的新手,在编写程序时通过不做规划,导致在最后的花费大量的时间才重新修正程序。所以编写程序的首要任务是做好规划,这样做可以减少之后的程序调试 (Debug) 时间。

一个良好算法程序的诞生需要经历以下几个阶段
\begin{enumerate}
	\item (Work an Example yourself) 自己做一个例子。
	\item (Write down what you just did) 记录你刚才的工作。
	\item (Generalize you steps) 总结前两个步骤中的工作。
	\item (Test you steps ) 测试结论。
	\item (Translate to code) 将结论转为代码形式。
	\item (Test you Program) 测试程序是否符合结论。
	\item (Debug your Program) 调试代码。

\end{enumerate}

\section{Compiliing (编译)}
编译程序就是将人类编写的代码变为计算机可执行的形式。

.c 文件编译后会生成.out文件,


\begin{figure}
	\centering
	\includegraphic{./figures/compileCFile.jpg}
	\label{fig:compileCFile}
	
\end{figure}

\subsection{C文件各部分介绍}
\subsubsection{头文件 (Header Files)}
头文件是指包含 \textbf{#include}的部分。
其中 \textbf{#include} 之后加 \textbf{<>} 是指:引入的文件是C语言的标准头文件 (头文件中通常包含三项内容:函数原型(function prototypes),宏定义(macro definitions)和类型声明(type declarations))。
若是之后加 \textbf{""} 是指:引入的文件是自己编写的头文件 (非标准头文件)


\cs{宏定义} 





\section{Testing and debugging}

\section{}
