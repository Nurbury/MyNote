\section{Type}

\subsection{进制}
常用的进制包括 \textbf{二进制、八进制、十进制、十六进制}

\begin{example}
	将 A (十进制) 转为十六进制.

	步骤一,使用 A 除 16 取其余数和商,
	步骤二,将商计入十六进制的结果中。
	步骤三,继续执行步骤一、二直至步骤一中的余数小于除数。
	步骤四,将余数计入结果。

\end{example}

\subsection{Basic Big Data}
以下是 C 语言所支持的数据类型。

\begin{table}[h]
	\textbf{type & size & interpretation & example} \\
	char & 1 byte & one ASCII character & 'f' \\
	int & 4 bytes & binary integer & 42 \\
	float & 4 bytes & floating point number & 3.14 \\
	double & 8 bytes & floating point number & 3.124124214124 \\

\end{table}

\subsection{int}
在C语言中的 int 类型有两种表示方式,1.unsigned 表示所有从0~$2^{32}$的值,2. signed 表示的一半正值一半负值。


\subsection{float & double}
float 拥有32bits用于表示单精度浮点数。

这32个bits可以划分为三个部分,其中最低的23 bits 对尾数 (mantissa) 编码,前面的8位对指数(幂,exponent)编码,最高位为符号位 s (s = 1, number = negative, s = 0, number = positive) 。 

double 拥有64bits 用于表示双精度浮点数。

这64bits中,52bits 用于对尾数编码,11bits用于对指数编码,最后1bit表示符号位。

\begin{figure}[h]
	\centering
	\includegraphics{../figures/floatBits.jpg}
	\label{fig:floatBtis}
\end{figure}



\subsection{类型转换 Type Conversion}
当不同的类型进行计算时,为了保证结果的精度,通常将结果使用参与计算的类型中最大的类型进行表示。

四种常见的转换方式:
\begin{enum}
\item 从较小的有符号整数类型转换为较长的有符号整数时,必须对数字进行符号扩展 - 必须将符号位(最高有效位)复制适当的次数以填充额外的位。
\item从较小的无符号整数类型转换为较长的无符号整数类型时,数字必须扩展为零 - 额外的位用所有零填充。
\item在自动转换期间可以更改位表示的第三种常见方式是将较长的整数类型转换为较短的整数类型。

\end{enum}

\subsection{溢出}
每一种类型都有它的范围,例如 \textbf{short} 类型,它的范围为 16bits 代表着它能够表示 $2^{16}$ 的数字。当超过这个范围,就会发生 \texttt{溢出} (signed short 会出现下溢,即32767 + 1 结果会变为-32768。而 signed short 会出现上溢, 即 32767 + 1 结果为 32768。)

\textbf{上溢 overflow} : 产生的数字太大,无法由原来的结果类型所表示。

\textbf{下溢 underflow} : 产生的数字太小,无法由原来的类型所表示。

\subsection{String}
String 是由一系列的character以及空终止符结尾 ('\0')的数据类型,

String 使用 \textbf{%s} 输出结果 。


\subsection{struct}
C语言的基本数据类型只有 char 、int 、float 、double 当我们需要使用其他类型(如String、Tree、Stuck) 时,必须使用 struck 进行定义。

结构体中的属性需要使用 "." 进行访问。

\subsection{枚举类型 enumerated}
枚举类型是命名常量的类型。

enum threat_level_t {
  LOW,
  GUARDED,
  ELEVATED,
  HIGH,
  SEVERE
};



